\documentclass{report}

\usepackage{xeCJK}
\setCJKmainfont{SimSun}

\usepackage{listings}
\lstset{breaklines,numbers=left}
\usepackage{hyperref}
\usepackage{titletoc}
\usepackage[titletoc]{appendix}

\title{Embedded System Design Practice Report II \\ Spring 2017}
\author{李约瀚 \\ 14130140331 \\ me@qinka.pro \\ qinka@live.com}

\begin{document}
    \maketitle
    \tableofcontents
    
    \chapter{License}
    \label{chap:license}
    This report and its codes are opened with GPL-3 license.
    
    The codes for these practices can be found on the%
    \href{https://github.com/Qinka/embedded-system-design-homework}{GitHub},
    and the ``codes'' include the source for this report(in \LaTeX).
    
    \chapter{Summary}
    \label{chap:summary}
    
    \chapter{Module Parameter \& Proc File System}
    \label{chap:mpnpsf}
    
    The second practice is about writing a module with parameter and create a proc file in the  proc file system.
    Use the parameter for module the setup the buffer's size, and use the proc file system to read and write the buffer.
    
    \section{Design}
    \label{chap:mpnpsf:design}
    
    So all I need to do are two things:  define a parameter, and create a file proc file system.
    For the first one, I use the ``standard'' macros to declare the parameter for buffer size.
    When it parameter of the buffer size not given, the default value(1024) will be used.
    For the second one, what I need to do are create a variable to store the data, and than initial and register the proc file.
    
    The macro \lstinline|module_param| is used to declare a variable to be parameter.
    And by using \lstinline|proc_create_data| the proc file can be create, and
    \lstinline|remove_proc_entry| can be used to remove a proc file.
    
    The details of source cam be found in listing \ref{code:led.c}.
        
        
    \section{Coding}
    \label{chap:mpnpsf:coding}
    
    The codes are similar with the ones in the practice I. The only difference between two
    practices is that the practice II uses proc file system, which lies in line 127 to line 141,
    in the initial and cleaning-up methods, at listing \ref{code:led.c}.
    
    \section{Testing}
    \label{chap:mpnpsf:test}
    
    For testing, the first step is load the module. I use `insmod' command to load the module.
    In this step, I can load the module with parameter \lstinline|buffer_size=2048|, 
    which will set up a 2048-byte buffer, and I can also load the module without any parameter.
    
    Then I launch the LEDs' deriver written in practice I, then run the testing just like
    what I run in the practice I.
    
        
    \begin{appendix}
        \chapter{Sources}
        \label{achap:source}
        
        In this section, there are the sources in the practices, while the hyper-link to GitHub will be given.
        
        \section{Module Parameter \& Proc File System}
        \label{src:mpnpsf}
        
        The following is the source's \href{https://github.com/Qinka/embedded-system-design-homework/blob/master/practice2/led.h}{header file} for module.
        \lstinputlisting[label={code:led.h},language=C,caption=Module Parameter \& Proc File System(led.h)]{practice2/led.h}
        
        The following is the \href{https://github.com/Qinka/embedded-system-design-homework/blob/master/practice2/led.c}{source}
        for module.
        \lstinputlisting[label={code:led.c},language=C,caption=Module Parameter \& Proc File System(led.)]{practice2/led.c}
        
        
        
    \end{appendix}
    
\end{document}