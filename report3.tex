\documentclass{report}

\usepackage{xeCJK}
\setCJKmainfont{SimSun}

\usepackage{listings}
\lstset{breaklines,numbers=left}
\usepackage{hyperref}
\usepackage{titletoc}
\usepackage[titletoc]{appendix}

\title{Embedded System Design Practice Report III \\ Spring 2017}
\author{李约瀚 \\ 14130140331 \\ me@qinka.pro \\ qinka@live.com}

\begin{document}
    \maketitle
    \tableofcontents
    
    \chapter{License}
    \label{chap:license}
    This report and its codes are opened with GPL-3 license.
    
    The codes for these practices can be found on the%
    \href{https://github.com/Qinka/embedded-system-design-homework}{GitHub},
    and the ``codes'' include the source for this report(in \LaTeX).
    
    \chapter{Summary}
    \label{chap:summary}
    
    \chapter{Tasklet \& Work-queue}
    \label{chap:tnw}
    
    The third practice is about writing timers and using tasklet and work-queue to
    execute tasks periodically.
    
    \section{Design}
    \label{chap:tnw:design}
    
    There are four tasks to do:
    \begin{itemize}
        \item Set up timers
        \item Set up tasklet
        \item Set up work-queue
        \item Write to LEDs device
    \end{itemize}

    Two timer are set up with different frequency, and when time up, the timers'
    handlers will schedule the tasklet and work-queue, and at the same time,
    the timers will be reset again.
    
    \section{Coding}
    \label{chap:tnw:coding}
    
    So in line 21 to line 28, at listing \ref{code:task.c}, the data are defined.
    In line 31 to line 32, the things about tasklet are defined,
    while in line 35 to line 37 the things about work-queue are defined.
    In line 41 to line 47, the timers are defined.
    In line 52 to line 65, the method to write to device is defined.
    In line 68 to line 90, there are the instances of workers of tasklet and work-queue.
    Finally, in line 93 to line 118, there are the initial and cleaning-up method
    for module, where initial and clean up the timers, tasklet, and work-queue.
    
    In line 52 to 65, the LEDs device is opened with method \lstinline|filp_open|.
    And by using the write operation in the struct, the string, which will control
    the LEDs, can be written in to device.
    
    In line 68 to 75, the workers for tasklet and work-queue write the control string
    into the device with the method \lstinline|write_dev|. For tasklet's worker,
    it writes string ``8888'' to let the one side of LED matrix light up,
    and meanwhile in the work-queue's worker, the string ``1111'' will be written into
    device, which will light up another side of LED matrix.
    
    In the line 95 to 107, at method of initial, in listing \ref{code:task.c},
    the data about tasklet and work-queue are filled up with what them should be.
    And the tasklet and work-queue are set up.
    Meanwhile, in the cleaning-up method of the module, the tasklet and work-queue will be cleaned up.

    \section{Testing}
    \label{sec:test}

    The testing is based on the practice I or practice II, because of using LED device.
    So the first step is load the module and driver in the practice I. Then load the 
    module of tasklet and work-queue.

    
    
    \begin{appendix}
        \chapter{Sources}
        \label{achap:source}
        
        In this section, there are the sources in the practices, while the hyper-link to GitHub will be given.
        
        \section{Tasklet \& Work-queue}
        \label{src:tnw}
        
        The following is the
        \href{https://github.com/Qinka/embedded-system-design-homework/blob/master/practice3/task.h}{header file}
        of source.
        \lstinputlisting[label={code:task.h},language=C,caption=Tasklet \& Work-queue(task.h)]{practice3/task.h}
        
        The following is the \href{https://github.com/Qinka/embedded-system-design-homework/blob/master/practice3/task.c}{source}.
        \lstinputlisting[label={code:task.c},language=C,caption=Tasklet \& Work-queue(task.c)]{practice3/task.c}
    \end{appendix}
    
\end{document}