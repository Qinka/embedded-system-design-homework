\documentclass{report}

\usepackage{xeCJK}
\setCJKmainfont{SimSun}

\usepackage{listings}
\lstset{breaklines,numbers=left}
\usepackage{hyperref}
\usepackage{titletoc}
\usepackage[titletoc]{appendix}

\title{Embedded System Design Practice Report I \\ Spring 2017}
\author{李约瀚 \\ 14130140331 \\ me@qinka.pro \\ qinka@live.com}

\begin{document}
    \maketitle
    \tableofcontents
    
    \chapter{License}
    \label{chap:license}
    This report and its codes are opened with GPL-3 license.
    
    The codes for these practices can be found on the%
    \href{https://github.com/Qinka/embedded-system-design-homework}{GitHub},
    and the ``codes'' include the source for this report(in \LaTeX).
    
    \chapter{Summary}
    \label{chap:summary}
    
    \chapter{Simple Character Device}
    \label{chap:scd}
    
    The first practice is about a simple character device. In this practice, I got the
    basic skill to create a character device. Meanwhile, I wrote a user-space device deriver for LED-matrix to test
    that character device.
    
    \section{Design}
    \label{chap:scd:design}
    
    Before writing the codes, I generally design the device:
    \begin{itemize}
        \item a global buffer \& counter
        \item open \& close method
        \item write \& read method
        \item others
    \end{itemize}
    
    For the global buffer, in the source I defined a counter and pointer for the buffer. When \lstinline|open()| executed,
    the counter will increase by 1; and when \lstinline|release()| executed, the counter will decrease by 1. 
    Meanwhile, in the methods -- \lstinline|open()| and \lstinline|close()|, the buffer will be initialed and 
    freed, by using method \lstinline|kmalloc()| and \lstinline|kfree()|. 
    In the \lstinline|write()| and \lstinline|read()|, the thing needed to do is writing or read buffer,
    and update the offset of pointer.
    
    The remaining things are register the character device to system, using
    \lstinline|alloc_chrdev_region|, and initial the device. The details can be found in listing \ref{code:led.c}.
    
    \section{Coding}
    \label{chap:scd:coding}
    
    In this section, I will talk about the sources for this deriver.
    In line 18 to line 24 at listing \ref{code:led.c}, the global variables are defined.
    In line 26 to line 142, the codes are about the file system operations.
    In line 147 to line 170, there is the initial method for kernel module, while
    the line 173 to line 182 is the method for cleaning up.
    
    In the open and close methods(line 26 ~ line 56, listing \ref{code:led.c}) for device,
    a count is set up, and it will increase or decrease when the device is opened or closed,
    while the buffer will automatically be allocated or freed, when counter is zero.
    In line 58 to line 76, the method to support system call -- \lstinline|lseek|.
    In the read and write methods(line 78 ~ 106, listing \ref{code:led.c}) for device,
    there are three thing to do:
    \begin{itemize}
        \item Bound checking
        \item Data coping
        \item Pointer setting
    \end{itemize}
    For the bound checking, the methods will check whether it is the end of buffer, and whether there are enough bytes to read or write. For the former, the function will return zero,
    while the function will reset the length of what user what to read or write, for the latter.
    In the line 108 to line 132, the codes is about ioctl. The functions in it include:
    \begin{itemize}
        \item free
        \item allocate
        \item reallocate
    \end{itemize}
    
    And in the Linux-4.4%
    \footnote{Till now, I do not know the version since which remove the operation \lstinline|ioctl|, and replace with \lstinline|unlocked_ioctl| and others operation.},
    the \lstinline|unlocked_ioctl| is available.
    
    In the initial and cleaning-up method, the method \lstinline|alloc_chrdev_region|
    to allocate major id for a character device, while method \lstinline|unregister_chrdev_region|
    to free it. Meanwhile, the \lstinline|cdev_init| is to initial a device, and the
    \lstinline|cdev_del| is to delete it.
        
    \section{Testing}
    \label{chap:scd:test}
    
    After the character device deriver finished, I wrote a user-space deriver for a LED matrix to
    light up the LEDs.
    The driver will read the character device, and parse the texts.
    The driver will scan the each line of matrix and using the first four lower bit in a character to control which LED will be lighted up. The detail of source can be found at listing \ref{code:d:daemon.c}.
    
    The load the these kernel module and deriver for LEDs, while the character device file 
    is created with command \lstinline|mknod|, and then using \lstinline|cat| or 
    \lstinline|echo| write the characters to buffer.
    
    
    
    \begin{appendix}
        \chapter{Sources}
        \label{achap:source}
        In this section, there are the sources in the practices, while the hyper-link to GitHub will be given.
        
        \section{Sources for Simple Character Device}
        \label{src:scd}
        
        The following is the source of%
        \href{https://github.com/Qinka/embedded-system-design-homework/blob/master/practice1/led.h}{the header file}.
        \lstinputlisting[label={code:led.h},language=C,caption=Simple Character Device Source(led.h)]{practice1/led.h}
        
        The following is the%
        \href{https://github.com/Qinka/embedded-system-design-homework/blob/master/practice1/led.c}{the source file}
        
        \lstinputlisting[label={code:led.c},language=C,caption=Simple Character Device Source(led.c)]{practice1/led.c}
        
        \section{User-space LEDs Matrix Driver}
        \label{src:uslmd}
        
        The following is the \href{https://github.com/Qinka/embedded-system-design-homework/blob/master/led_daemon/daemon.c}{source} of a user-space driver for a $4 \times 4$ .
        \lstinputlisting[label={code:d:daemon.c},language=C,caption=User-space LEDs Matrix Drivex]{led_daemon/daemon.c}
         
    \end{appendix}
    
    
\end{document}